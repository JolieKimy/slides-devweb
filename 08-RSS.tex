\hypertarget{syndication}{%
\section{Syndication}\label{syndication}}

\begin{itemize}
\tightlist
\item
  Principe de vendre un contenu à plusieurs médias
\item
  Dans les journaux : dépêches, bandes dessinées, \ldots{}
\item
  Télévision : jeux, séries
\item
  Web : Flux RSS / Atom

  \begin{itemize}
  \tightlist
  \item
    1 source de donnée, plusieurs abonnés
  \item
    Contenu : news, blogs, podcast, \ldots{}
  \item
    Accès unique à plusieurs sources d'informations
  \item
    Mises à jour fréquentes
  \end{itemize}
\end{itemize}

\hypertarget{historique}{%
\section{Historique}\label{historique}}

\begin{itemize}
\tightlist
\item
  Feed (fil ou flux) RSS
\item
  Format d'échange de données en XML

  \begin{itemize}
  \tightlist
  \item
    fournir ou recueillir des données structurées
  \end{itemize}
\item
  Utilisation d'un lecteur (agrégateur) RSS
\item
  RSS V.90 Créé en 1999 par Netscape
\item
  RSS v1.0 par O'Reilly en 2000
\item
  RSS v2.0 par Dave Winer (Harvard) en 2002
\item
  Atom v1.0 en 2005 (développement communautaire)
\end{itemize}

Il y a \emph{neuf} versions de RSS généralement incompatibles entre
elles. Lire
\href{https://web.archive.org/web/20110726001954/http://diveintomark.org/archives/2004/02/04/incompatible-rss}{The
myth of RSS compatibility}

\hypertarget{applications}{%
\section{Applications}\label{applications}}

\begin{itemize}
\tightlist
\item
  Récupérer l'info pour :

  \begin{itemize}
  \tightlist
  \item
    la lire
  \item
    la réutiliser sur un site
  \end{itemize}
\item
  News
\item
  Notification : activité, mise à jour
\item
  Podcasts
\item
  Accès unique à des infos de plusieurs sites
\item
  Source de contenu
\item
  Augmenter le trafic d'un site
\item
  \href{http://blog.louisgray.com/2008/11/30-different-uses-for-rss.html}{Exemples}
  et
  \href{http://www.makeuseof.com/tag/14-other-ways-to-use-rss-feeds/}{Passerelles}
\end{itemize}

\hypertarget{agruxe9gateurs}{%
\section{Agrégateurs}\label{agruxe9gateurs}}

\begin{itemize}
\tightlist
\item
  Natifs

  \begin{itemize}
  \tightlist
  \item
    Navigateurs (IE, FF, \ldots{})
  \item
    Clients mail (OL, TB, Evolution, \ldots{})
  \item
    Applis dédiées (Newsgator, FeedDemon, \ldots{})
  \end{itemize}
\item
  WebApps

  \begin{itemize}
  \tightlist
  \item
    Feedly, NetVibes, Sniptracker\ldots{}
  \end{itemize}
\item
  Extensions

  \begin{itemize}
  \tightlist
  \item
    Sage
  \end{itemize}
\item
  \href{https://en.wikipedia.org/wiki/Comparison_of_feed_aggregators}{Liste}
\end{itemize}

\hypertarget{guxe9nuxe9rer-un-flux-rss}{%
\section{Générer un flux RSS}\label{guxe9nuxe9rer-un-flux-rss}}

\begin{itemize}
\tightlist
\item
  Fichier XML :

  \begin{itemize}
  \tightlist
  \item
    Canal / Items (RSS)
  \item
    Entrées (Atom)
  \end{itemize}
\item
  Indiquer le flux au navigateur
\item
  Permettre l'abonnement : logo visible dans la page
\item
  Génération dynamique du fichier XML
\end{itemize}

\hypertarget{formats}{%
\section{Formats}\label{formats}}

\begin{itemize}
\tightlist
\item
  RSS 2.0 (Really Simple Syndication)

  \begin{itemize}
  \tightlist
  \item
    Simple, le plus répandu
  \item
    Extensible par modules (éléments supplémentaires)
  \end{itemize}
\item
  Atom 1.0 : 2 standards web

  \begin{itemize}
  \tightlist
  \item
    Atom Syndication Format
  \item
    Atom Publishing Protocol
  \end{itemize}
\item
  RSS 0.90, 1.0 (RDF Site Summary) : obsolète

  \begin{itemize}
  \tightlist
  \item
    Basé sur RDF
  \item
    Extensible par modules
  \end{itemize}
\item
  Antérieurs : RSS 0.91, 0.92 (Rich Site Summary) : obsolètes

  \begin{itemize}
  \tightlist
  \item
    Migration facile vers RSS 2.0
  \end{itemize}
\end{itemize}

\hypertarget{rss-2.0}{%
\section{RSS 2.0}\label{rss-2.0}}

\begin{otherlanguage}{english}

\begin{Shaded}
\begin{Highlighting}[]
\KeywordTok{<?xml}\NormalTok{ version="1.0" encoding="utf-8"}\KeywordTok{?>}
\KeywordTok{<rss}\OtherTok{ version=}\StringTok{"2.0"}\KeywordTok{>}
    \KeywordTok{<channel>}
    
        \KeywordTok{<title>}\NormalTok{Arc Info News RSS 2.0}\KeywordTok{</title>}
        \KeywordTok{<link>}\NormalTok{http://www.he-arc.ch/}\KeywordTok{</link>}
        \KeywordTok{<description>}\NormalTok{News HE-Arc (RSS 2.0)}\KeywordTok{</description>}
        
        \KeywordTok{<language>}\NormalTok{fr}\KeywordTok{</language>}
        \KeywordTok{<pubDate>}\NormalTok{Sun, 26 Oct 2008 04:00:00 GMT}\KeywordTok{</pubDate>}
        \KeywordTok{<lastBuildDate>}\NormalTok{Sun, 26 Oct 2008 09:41:01 GMT}\KeywordTok{</lastBuildDate>}
        \KeywordTok{<docs>}\NormalTok{http://blogs.law.harvard.edu/tech/rss}\KeywordTok{</docs>}
        \KeywordTok{<managingEditor>}\NormalTok{david.grunenwald@he-arc.ch}\KeywordTok{</managingEditor>}
        \KeywordTok{<webMaster>}\NormalTok{david.grunenwald@he-arc.ch}\KeywordTok{</webMaster>}
        \KeywordTok{<ttl>}\NormalTok{5}\KeywordTok{</ttl>}
    
        \KeywordTok{<item>}
          \KeywordTok{<title>}\NormalTok{Nouveau cours d'Applications Internet 2}\KeywordTok{</title>}
          \KeywordTok{<link>}\NormalTok{https://intranet.he-arc.ch/sites/ingenierie/}
\NormalTok{          Bachelor_Modules_Annees_Fich/12-13/Niveau-3/}
\NormalTok{          ING-DM3254-12-D%C3%A9veloppement%20web%20et%20mobile-V1.docx}\KeywordTok{</link>}
          \KeywordTok{<description>}\NormalTok{Un nouveau cours}\KeywordTok{</description>}
          \KeywordTok{<pubDate>}\NormalTok{Mon, 27 Oct 2008 09:39:21 GMT}\KeywordTok{</pubDate>}
        \KeywordTok{</item>}
    
    \KeywordTok{</channel>}
\KeywordTok{</rss>}
\end{Highlighting}
\end{Shaded}

\end{otherlanguage}

\hypertarget{atom-1.0}{%
\section{Atom 1.0}\label{atom-1.0}}

\begin{otherlanguage}{english}

\begin{Shaded}
\begin{Highlighting}[]
\KeywordTok{<?xml}\NormalTok{ version="1.0" encoding="utf-8"}\KeywordTok{?>}
\KeywordTok{<feed}\OtherTok{ xmlns=}\StringTok{"http://www.w3.org/2005/Atom"}\KeywordTok{>}
 
    \KeywordTok{<title>}\NormalTok{Arc Info News Atom 1.0}\KeywordTok{</title>}
    \KeywordTok{<subtitle>}\NormalTok{version Atom}\KeywordTok{</subtitle>}
    \KeywordTok{<link}\OtherTok{ rel=}\StringTok{"self"}\OtherTok{ type=}\StringTok{"application/atom+xml"} 
\OtherTok{        href=}\StringTok{"http://www.he-arc.ch/rss-generator/atom.php"} \KeywordTok{/>}

    \KeywordTok{<updated>}\NormalTok{2008-10-27T18:30:02Z}\KeywordTok{</updated>}
    \KeywordTok{<author>}
        \KeywordTok{<name>}\NormalTok{David Grunenwald}\KeywordTok{</name>}
        \KeywordTok{<email>}\NormalTok{david.grunenwald@he-arc.ch}\KeywordTok{</email>}
    \KeywordTok{</author>}
    \KeywordTok{<id>}\NormalTok{http://dgr.he-arc.ch/}\KeywordTok{</id>}
     
    \KeywordTok{<entry>}
        \KeywordTok{<title>}\NormalTok{Nouveau cours d'Applications Internet 2}\KeywordTok{</title>}
        \KeywordTok{<link>}\NormalTok{https://intranet.he-arc.ch/sites/ingenierie/}
\NormalTok{          Bachelor_Modules_Annees_Fich/12-13/Niveau-3/}
\NormalTok{          ING-DM3254-12-D%C3%A9veloppement%20web%20et%20mobile-V1.docx}\KeywordTok{</link>}
        \KeywordTok{<id>}\NormalTok{http://dgr.he-arc.ch/atom/1234}\KeywordTok{</id>}
        \KeywordTok{<updated>}\NormalTok{2008-10-27T18:30:02Z}\KeywordTok{</updated>}
        \KeywordTok{<summary>}\NormalTok{Un tout nouveau cours.}\KeywordTok{</summary>}
    \KeywordTok{</entry>}
 
\KeywordTok{</feed>}
\end{Highlighting}
\end{Shaded}

\end{otherlanguage}

\hypertarget{guxe9nuxe9rer-le-flux}{%
\section{Générer le flux}\label{guxe9nuxe9rer-le-flux}}

\begin{itemize}
\tightlist
\item
  Données dynamiques
\item
  Source de données identique à celle de l'application
\item
  Nécessité de générer le fichier à la volée
\item
  Nouveaux items en premier
\item
  Possibilité d'afficher le flux avec XSLT
\end{itemize}

\hypertarget{signaler-la-pruxe9sence-dun-fil-rss}{%
\section{Signaler la présence d'un fil
RSS}\label{signaler-la-pruxe9sence-dun-fil-rss}}

\begin{itemize}
\tightlist
\item
  Au navigateur
\end{itemize}

\begin{otherlanguage}{english}

\begin{Shaded}
\begin{Highlighting}[]
\KeywordTok{<link}\OtherTok{ rel=}\StringTok{"alternate"}\OtherTok{ type=}\StringTok{"application/rss+xml"}\OtherTok{ title=}\StringTok{"RSS"} 
\OtherTok{    href=}\StringTok{"http://www.site.tld/feedfilename.xml"}\KeywordTok{>}
\end{Highlighting}
\end{Shaded}

\end{otherlanguage}

\begin{itemize}
\tightlist
\item
  À l'utilisateur

  \begin{itemize}
  \tightlist
  \item
    Icône + lien vers le script générant le flux
  \end{itemize}

  \begin{otherlanguage}{english}

\begin{Shaded}
\begin{Highlighting}[]
\KeywordTok{<a}\OtherTok{ href=}\StringTok{"http://www.site.tld/feed"}\KeywordTok{>}
\KeywordTok{<img}\OtherTok{ src=}\StringTok{"rss-icon.png"}\OtherTok{ alt=}\StringTok{"M'abonner"} \KeywordTok{/></a>} 
\end{Highlighting}
\end{Shaded}

  \end{otherlanguage}
\item
  Valider un flux

  \begin{itemize}
  \tightlist
  \item
    \href{http://validator.w3.org/feed/}{w3c}
  \item
    \href{http://www.feedvalidator.org/}{feedvalidator}
  \end{itemize}
\item
  MIME Types

  \begin{itemize}
  \tightlist
  \item
    application/atom+xml
  \item
    application/rss+xml
  \end{itemize}
\end{itemize}

\hypertarget{podcasts}{%
\section{Podcasts}\label{podcasts}}

\begin{itemize}
\tightlist
\item
  Elément en RSS 2.0 :
\end{itemize}

\begin{otherlanguage}{english}

\begin{Shaded}
\begin{Highlighting}[]
\KeywordTok{<item>}
\KeywordTok{<title>}\NormalTok{Podcast}\KeywordTok{</title>}
\KeywordTok{<link>}\NormalTok{http://www.website_url.com}\KeywordTok{</link>}
\KeywordTok{<description>}\NormalTok{Podcast : audio.mp3}\KeywordTok{</description>}

\KeywordTok{<enclosure}\OtherTok{ url=}\StringTok{"http://www.site.tld/sounds/audio.mp3"}\OtherTok{ length=}\StringTok{"666666"}\OtherTok{ type=}\StringTok{"audio/mpeg"}\KeywordTok{/>}

\KeywordTok{<guid}\OtherTok{ isPermaLink=}\StringTok{"false"}\KeywordTok{>}\NormalTok{2004-11-30-02}\KeywordTok{</guid>}
\KeywordTok{</item>}
\end{Highlighting}
\end{Shaded}

\end{otherlanguage}

\begin{itemize}
\tightlist
\item
  Elément en Atom 1.0 :
\end{itemize}

\begin{otherlanguage}{english}

\begin{Shaded}
\begin{Highlighting}[]
\KeywordTok{<entry>} 
\KeywordTok{<id>}\NormalTok{http://www.example.org/entries/1}\KeywordTok{</id>} \KeywordTok{<title>}\NormalTok{Atom 1.0}\KeywordTok{</title>} 
\KeywordTok{<updated>}\NormalTok{2005-07-15T12:00:00Z}\KeywordTok{</updated>} 
\KeywordTok{<link}\OtherTok{ href=}\StringTok{"http://www.example.org/entries/1"} \KeywordTok{/>} 
\KeywordTok{<summary>}\NormalTok{An overview of Atom 1.0}\KeywordTok{</summary>} 

\KeywordTok{<link}\OtherTok{ rel=}\StringTok{"enclosure"} 
\OtherTok{type=}\StringTok{"audio/mpeg"}\OtherTok{ title=}\StringTok{"Sttellla - ça va comme un lundi"} 
\OtherTok{href=}\StringTok{" http://www.site.tld/sounds/audio.mp3 "}
\OtherTok{length=}\StringTok{"666666"} \KeywordTok{/>}

\KeywordTok{</entry>}
\end{Highlighting}
\end{Shaded}

\end{otherlanguage}

\hypertarget{alternatives}{%
\section{Alternatives}\label{alternatives}}

\begin{itemize}
\tightlist
\item
  \href{http://ogp.me/}{Facebook Open Graph}
\item
  \href{https://dev.twitter.com/cards/overview}{Twitter Cards}
\item
  \href{http://schema.org/}{Google Schema.org}
\item
  \href{http://microformats.org/}{Microformats}
\item
  \href{http://json-ld.org/}{JSON-LD}
\end{itemize}

De multiples spécifications permettent d'enrichir le contenu d'une page
afin de la rendre aisément « consommable » par un moteur de recherche,
ou une plateforme sociale (e.g.~Facebook, Twitter, Reddit, etc.)

RDF/XML (utilisé par RSS 0.90, 0.91) est progressivement remplacé par
les \emph{microdata} (Schema.org), RDFa ou JSON-LD. Les microformats
sont notamment utilisés par LinkedIn.

\hypertarget{pour-en-savoir-plus}{%
\section{Pour en savoir plus\ldots{}}\label{pour-en-savoir-plus}}

\begin{itemize}
\tightlist
\item
  \href{http://www.xul.fr/xml-rss.html}{Étapes de création d'un flux}
\item
  \href{http://www.rssboard.org/rss-specification}{Spécification RSS
  2.0}
\item
  \href{https://tools.ietf.org/html/rfc4287}{Spécification Atom 1.0}
\item
  \href{http://www.differencebetween.info/difference-between-rss-and-atom}{Comparatif
  RSS 2.0 / Atom 1.0}
\item
  \href{https://trends.builtwith.com/feeds}{Stats} d'utilisation
\item
  \href{http://www.makeuseof.com/tag/rss-dead-look-numbers/}{Is RSS dead
  ?} (03.2015)
\end{itemize}

\begin{otherlanguage}{english}

\end{otherlanguage}

\begin{otherlanguage}{english}

\end{otherlanguage}

\begin{otherlanguage}{english}

\end{otherlanguage}

\hypertarget{sources}{%
\section{Sources}\label{sources}}
