\hypertarget{programme}{%
\section{Programme}\label{programme}}

\begin{itemize}
\tightlist
\item
  Frameworks MVC : Laravel, (Rails), Django, \ldots{}
\item
  HTML5 : vue d'ensemble
\item
  Javascript : AJAX, DOM, JSON, Node.js, jQuery
\item
  (Syndication : RSS, Atom)
\item
  Déploiement et configuration Serveur
\item
  (Responsive) Web Design
\item
  Webservices : REST vs SOAP
\item
  Sécurité : Technologies, prévention des risques courants
\item
  {Vos souhaits ?}
\item
  Slides cours :
  \href{https://he-arc.github.io/slides-devweb/}{ghpages},
  \href{https://github.com/HE-Arc/slides-devweb/tree/master/src}{Source
  : github/HE-Arc}
\end{itemize}

\hypertarget{organisation}{%
\section{Organisation}\label{organisation}}

\begin{itemize}
\tightlist
\item
  Cours
\item
  Workshops intervenants externes

  \begin{itemize}
  \tightlist
  \item
    Automatisation du déploiement (R. Emourgeon) ?
  \item
    Flask (M. Amiguet) en janvier 2019
  \item
    Webdesign (M. Schmalstieg) ?
  \item
    Vue.js ? React ? AngularJS ?
  \end{itemize}
\item
  2 Projets

  \begin{itemize}
  \tightlist
  \item
    2 frameworks : Laravel \& Django (ouvert à d'autres propositions)
  \item
    Groupes de 3,
    \href{https://www.he-arc.ch/sites/www.he-arc.ch/files/Reglements/04\%20Formation\%20de\%20base/43\%20Ing\%C3\%A9nierie/430.100\%20Descriptifs\%20de\%20modules\%20Informatique/RS430.100.18.3255\%20Technologies\%20d'interaction.pdf}{30h}
    par personne et par projet
  \item
    Présentation de 20min
  \end{itemize}
\item
  Vos présentations ? {Vos propositions ?}
\end{itemize}

\hypertarget{projets}{%
\section{Projets}\label{projets}}

\begin{itemize}
\tightlist
\item
  Faire pour apprendre
\item
  Les rôles dans une équipe de développement web
\item
  Ne pas réinventer la roue ou tout faire soi-même
\item
  Critères d'évaluation d'un projet
\item
  En profiter pour apprendre des choses qui vous intéressent
\item
  Avant le 1er octobre :

  \begin{itemize}
  \tightlist
  \item
    Avoir un compte github avec une
    \href{https://github.com/settings/keys}{clé SSH} (indispensable au
    déploiement)
  \item
    Constitution des équipes de 3 personnes
  \item
    Choix du projet
  \item
    Forge : Créer projet sur github dans l'entité
    \href{https://github.com/HE-Arc/}{HE-Arc}
  \item
    \href{https://github.com/HE-Arc/slides-devweb/wiki/Projets-2018-2019}{S'inscrire}
  \end{itemize}
\item
  Offre d'essai Pluralsight 90 jours sur
  \href{https://imagine.microsoft.com/fr-fr/Catalog/Product/21}{MS
  Imagine}
\end{itemize}

\hypertarget{choix-des-projets}{%
\section{Choix des projets}\label{choix-des-projets}}

\begin{itemize}
\tightlist
\item
  Contrainte : appli basée sur des données
\item
  Choix

  \begin{itemize}
  \tightlist
  \item
    Besoin réel (ex: Concours robots P1 TIN)
  \item
    Données existantes :
    \href{https://inventory.ing.he-arc.ch/}{Inventaire},
    \href{http://wiki.dbpedia.org/}{dbpedia},
    \href{https://opendata.swiss/fr/}{opendata}, DB Bikini Test à dispo
  \item
    S'inspirer de l'existant :

    \begin{itemize}
    \tightlist
    \item
      \href{https://www.producthunt.com/topics/web-app}{Product Hunt},
      \href{http://blinklist.com/reviews}{blinklist},
      \href{http://www.makeuseof.com/tag/best-websites-internet/}{makeuseof},
      \ldots{}
    \item
      Volées précédentes : \href{https://he-arc.github.io/}{2016-18},
      \href{https://projets-labinfo.he-arc.ch/projects/webdev/wiki/Wiki\#Projets-2015-2016}{2015/16},
      \href{https://forge.ing.he-arc.ch/projects/1415-dw/wiki/Wiki}{2014/15}
    \end{itemize}
  \end{itemize}
\end{itemize}

\hypertarget{calendrier}{%
\section{Calendrier}\label{calendrier}}

\begin{longtable}[]{@{}rlrl@{}}
\toprule
Semaine & Automne & Semaine & Printemps\tabularnewline
\midrule
\endhead
38 & & 8 &\tabularnewline
39 & Projet Laravel & 9 &\tabularnewline
40 & & 10 &\tabularnewline
41 & & 11 &\tabularnewline
43 & & 12 &\tabularnewline
44 & & 13 &\tabularnewline
45 & & 14 &\tabularnewline
46 & & 15 & Présentations\tabularnewline
47 & T. Autonome & 17 & Présentations\tabularnewline
48 & & 18 &\tabularnewline
49 & & 19 & T. Autonome\tabularnewline
50 & Présentations & 20 & Examens\tabularnewline
51 & Présentations & 21 & Début TB\tabularnewline
2 & & &\tabularnewline
3 & Projet Python & &\tabularnewline
4 & & &\tabularnewline
5 & T. Autonome & &\tabularnewline
6 & Examen & &\tabularnewline
\bottomrule
\end{longtable}

\hypertarget{jalons-objectifs-uxe0-atteindre-pour-le-duxe9but-de-la-semaine}{%
\section{Jalons (Objectifs à atteindre pour le début de la
semaine)}\label{jalons-objectifs-uxe0-atteindre-pour-le-duxe9but-de-la-semaine}}

\begin{itemize}
\tightlist
\item
  1
\item
  2 Objectifs et maquettes
\item
  3 Authentification et 1er déploiement
\item
  4
\item
  5 Modèles avec relations (au moins 3)
\item
  6
\item
  7 Minimal Viable Product
\item
  8
\item
  9
\item
  10
\item
  11 Rendu projet, Présentation
\end{itemize}

\hypertarget{conseils}{%
\section{Conseils}\label{conseils}}

\begin{itemize}
\tightlist
\item
  Le plus simple possible
\item
  Pas trop de données
\item
  Application crédible (vraies données, cas réalistes)
\item
  Projet à blanc pour la prise en main du framework
\item
  \href{https://brainhub.eu/blog/difference-between-wireframe-mockup-prototype/}{Maquettes}
\item
  \href{http://drewfradette.ca/a-simpler-successful-git-branching-model/}{Organisez}
  l'utilisation du dépôt
\item
  Le temps disponible à l'horaire ne suffira pas !
\item
  Essayez de commit avec la même identité
\item
  Signalez dans le commit msg si vous n'êtes pas l'auteur
\item
  Le déploiement est long : commencez tôt !
\item
  Il est moins risqué travailler plus au début du projet qu'à la fin !
\end{itemize}

\hypertarget{uxe9valuation}{%
\section{Évaluation}\label{uxe9valuation}}

\begin{itemize}
\tightlist
\item
  User Experience : 50\%

  \begin{itemize}
  \tightlist
  \item
    Utilisabilité : Efficacité, efficience, satisfaction
  \item
    Design UI
  \end{itemize}
\item
  Code : 30\%

  \begin{itemize}
  \tightlist
  \item
    Absence bugs, qualité code, lisibilité
  \item
    Respect conventions et bonnes pratiques
  \item
    Déploiement, configuration
  \end{itemize}
\item
  Gestion de projet : 20\%

  \begin{itemize}
  \tightlist
  \item
    Fichiers versionnés, messages de commit
  \item
    Issues, planification, travail en équipe
  \item
    Documentation (wiki)
  \item
    Investissement, volume de travail
  \end{itemize}
\item
  Bonus (ceux qui vont plus loin) : 0-20\%

  \begin{itemize}
  \tightlist
  \item
    WebSockets ou autre API HTML5,
  \item
    WebService, \ldots{}
  \end{itemize}
\item
  Tous les membres d'un groupe n'ont pas forcément la même note
\end{itemize}

\hypertarget{pruxe9sentation-facultative}{%
\section{Présentation facultative}\label{pruxe9sentation-facultative}}

\begin{itemize}
\tightlist
\item
  Facultatif, ne peut qu'augmenter la moyenne
\item
  DOIT être annoncé au semestre d'automne
\item
  Un thème absent du cours
\item
  2 à 4 personnes
\item
  Une présentation claire avec démo (printemps)
\item
  Un exercice d'application
\item
  Critiques et discussion
\item
  Au plus tôt :

  \begin{itemize}
  \tightlist
  \item
    Constitution des équipes
  \item
    Proposer 1 à 3 thèmes
  \item
    \href{https://docs.google.com/spreadsheet/viewform?formkey=dEVJRE1WVTVPelhFcE94TGF5N1c0cGc6MQ}{Proposer}
    le(s) thème(s) de présentation et l'équipe
  \end{itemize}
\end{itemize}

\hypertarget{mon-expuxe9rience-en-duxe9veloppement-web}{%
\section{Mon expérience en développement
web}\label{mon-expuxe9rience-en-duxe9veloppement-web}}

\begin{itemize}
\tightlist
\item
  \href{https://docs.google.com/spreadsheet/viewform?formkey=dDg5Znh5akRBV1hPbC1qYlVRV3BONFE6MQ}{Questionnaire}
  obligatoire (votre username github vous y sera demandé)
\end{itemize}

\hypertarget{m-e-r-c-i}{%
\subsubsection{M E R C I !}\label{m-e-r-c-i}}

\begin{otherlanguage}{english}

\end{otherlanguage}

\begin{otherlanguage}{english}

\end{otherlanguage}

\begin{otherlanguage}{english}

\end{otherlanguage}

\hypertarget{sources}{%
\section{Sources}\label{sources}}
